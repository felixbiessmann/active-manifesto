%%!TEX TS-program = latex
\documentclass[]{beamer}
%\usepackage{helvet}
%\usepackage{pstricks,pst-node,pst-tree}
\usepackage{graphicx}
\hypersetup{pdfpagemode=FullScreen}
\usetheme{Singapore}
%\usetheme{copenhagen}
%\usetheme{Boadilla}
%\usetheme{Warsaw} 
\usecolortheme{seagull} 
\setbeamercovered{transparent}
\beamertemplatenavigationsymbolsempty 
\setbeamertemplate{footline}[frame number]

\usepackage{hyperref}
\usepackage{amsmath}
\usepackage{amsfonts}
\usepackage{amssymb}
\usepackage{booktabs}
\usepackage{dsfont}
\usepackage{multicol}
\usepackage{multirow}

\newcommand{\R}{\ensuremath{\mathds{R}}}
\newcommand{\C}{\ensuremath{\mathds{C}}}
\newcommand{\Q}{\ensuremath{\mathds{Q}}}
\newcommand{\N}{\ensuremath{\mathcal{N}}}
\newcommand{\Z}{\ensuremath{\mathds{Z}}}

\setbeamerfont{bib}{size*={4.00}{4.00}}
\usepackage[square, sort, numbers, authoryear]{natbib}
\renewcommand{\bibsection}{\subsubsection*{\bibname } }


\DeclareMathOperator*{\argmax}{argmax}
\DeclareMathOperator*{\argmin}{argmin}
\DeclareMathOperator*{\Corr}{Corr}
\DeclareMathOperator*{\E}{E}
\DeclareMathOperator*{\sign}{sign}
\renewcommand{\vec}[1]{\mathbf{#1}}
\usepackage{hyperref}

\usepackage{multicol}
\usepackage{multirow}
\usepackage{pbox}


\institute[]{}
%\logo{\pgfimage[width=.8cm,height=.8cm]{../KU_logo}}
\title[Active Manifesto]{
{
Speeding up the Manifesto Project: \\ Active learning strategies for \\efficient automated political annotations
}}
\author{
Felix Biessmann\thanks{felix.biessmann@gmail.com},~ 
Philipp Schmidt\thanks{schmidtiphil@gmail.com}
}
\date{}
\begin{document}

\begin{frame} 
\titlepage 
\end{frame}	

%
\section{Intro}
\subsection{}

\begin{frame}\frametitle{Disclaimers}
\small
\begin{itemize}
\item (For us) This open source project is a hobby
\item It has nothing to do with our job
\item Apologies if we missed to cite somebody in this room
\item We'd be excited to hear about more related work
\end{itemize}
\end{frame}

\begin{frame}\frametitle{Motivation}
\begin{itemize}[<+->]
\item Automated political analysis required for 
\begin{itemize}
\item Political scientists
\item Journalists
\item Average media consumer
\end{itemize}
\item ML models need in-domain training data \footnote{\cite{Biessmann16}}
\item But annotation budget is often limited:
\begin{itemize}
\item Temporal constraints (before elections) \footnote{\cite{merz2017}, \cite{bronline}}
\item Online news media (too much content) 
\end{itemize}
\item[$\rightarrow$] How to choose which texts to annotate?
\end{itemize}
\end{frame}

\begin{frame}\frametitle{Active Learning}
\begin{itemize}[<+->]
\item Given limited annotation budget, find the best model
\item How? \\
\begin{itemize}
\item Annotate difficult ones\footnote{For which model is most uncertain.} first
\end{itemize}
\item Why?
\begin{itemize}
\item Intuition: \\
\textit{ Model learns most from difficult examples}\\
\item Math:\\
\textit{ Gradient of loss function is larger for difficult examples }\\
\end{itemize}
\end{itemize}
\end{frame}

\section{Methods}
\subsection{}

\begin{frame}\frametitle{Data}

\begin{itemize}
\item All annotated German texts from:\\ 
\url{https://manifestoproject.wzb.eu/} 
\item Custom python tooling for manifesto API:\\
{\footnotesize \url{https://github.com/felixbiessmann/active-manifesto} }\\
\item Only texts with more than 1000 observed labels
\end{itemize}
\end{frame}

\begin{frame}\frametitle{Model}
\begin{itemize}
\item Preprocessing
\begin{itemize}
\item Unigram Bag-of-Words features
\item Hashing Vectorizer
\end{itemize}
\item Classification Model: Multinomial Logistic Regression
\begin{eqnarray}\label{eq:logreg_multiclass}
p(y=k|\vec{x}) = \frac{e^{z_k}}{\sum_{j=1}^K e^{z_j}}  \textrm{ with }  z_k=\vec{w}_k^{\top}\vec{x}.
\end{eqnarray}
With
\begin{itemize}
\item Labels $y\in\{1,2,\dots,K\}$ (manifesto code)\\ 
\item $\vec{w}_1,\dots,\vec{w}_K\in\R^{d}$ weight vectors of $k$th manifesto code\\ 
%\item $L_2$ norm regularization of weights
\end{itemize}
\end{itemize}
\end{frame}

\begin{frame}\frametitle{Offline Experiments}

\begin{itemize}
\item Train model on 1\%, 10\%, 20\%, \dots, 100\% of training data
\item Vary sampling strategies to select from unlabelled texts
\item Compute accuracy on hold-out data
\end{itemize}
\end{frame}

\begin{frame}\frametitle{Active Learning Strategies}

\begin{itemize}
\item Random Baseline: Uniform random sampling
\item Uncertainty Sampling: Only top-prediction counts
\begin{align}\label{eq:uncertainty_sampling}
\vec{x}_i = \argmax_{i,k} \left(1- p(y=k|\vec{x}_i,\vec{W})\right)
\end{align}
\item Entropy Sampling: All predictions count
\begin{align}\label{eq:entropy_sampling}
\vec{x}_i = \argmax_{i} \sum_k p(y=k|\vec{x}_i,\vec{W}) \log(p(y=k|\vec{x}_i,\vec{W}))
\end{align}
\item Margin Sampling: Top 2 predictions count
\begin{align}\label{eq:entropy_sampling}
\vec{x}_i = \argmin_{i} \left(p(y=k_1|\vec{x}_i,\vec{W}) - p(y=k_2|\vec{x}_i,\vec{W}) \right)
\end{align}
\end{itemize}
\end{frame}




\section{Results}
\subsection{}


\begin{frame}\frametitle{Results: 'Perfect' Reference Model}
\footnotesize
\begin{table}
\centering
\begin{tabular}{cccccc}
\toprule
&  manifesto code & precision  &  recall&  f1-score &  support\\
\midrule
&   107&  0.60& 0.48& 0.53&  774\\
&   201&  0.51& 0.55& 0.53& 1194\\
&   202&  0.63& 0.57& 0.60&  983\\
&   305&  0.46& 0.59& 0.52&  783\\
&   403&  0.52& 0.48& 0.50& 1281\\
&   411&  0.39& 0.60& 0.47& 1535\\
&   501&  0.61& 0.55& 0.58& 1380\\
&   502&  0.65& 0.41& 0.50&  587\\
&   503&  0.46& 0.52& 0.49& 2083\\
&   506&  0.63& 0.48& 0.54& 1026\\
&   605&  0.56& 0.44& 0.49&  576\\
&   701&  0.59& 0.39& 0.47& 1123\\
\bottomrule
& avg / total&  0.50& 0.48& 0.48&17559\\
\end{tabular}
\caption{Precision, recall, F1 score and number of instances per class. }
\label{tab:baseline_model_report} 
\end{table}

\end{frame}

\begin{frame}\frametitle{Active Learning Results}
\begin{center}
\includegraphics[width=\textwidth]{images/active_learning_manifesto.pdf} \\
Median accuracy and the 5th/95th percentile across 100 repetitions \\
\end{center}

\end{frame}

\section{Conclusion}
\subsection{}


\begin{frame}\frametitle{Conclusion}
\begin{itemize}
\item Automated political analysis requires annotations
\item \textbf{Limited budged} for annotations of political texts 
\item Active Learning
\begin{itemize}
\item Helps to select which texts to annotate
\item Perfect model with 80\% of data
\item Almost perfect (over 90\%) with 50\% of data
\end{itemize}
\item[$\rightarrow$] Active learning can speed up political annotations. 
\item Code:\\
\footnotesize
 \url{https://github.com/felixbiessmann/active-manifesto} 
\end{itemize}
\end{frame}

\begin{frame}\frametitle{Limitations}
\begin{itemize}
\item We used a simple model
\item Random sampling is an unrealistic baseline
\item We only performed offline experiments
\item[$\rightarrow$] More convincing: online experiments
\end{itemize}
\end{frame}


\section{Demo}
\subsection{}


\begin{frame}\frametitle{Demo \url{http://rightornot.info}}
\begin{itemize}
\item Goal: Collect Annotations with Active Learning
\begin{enumerate}
\item For political analysis of non-manifesto texts
\item For comparing manifesto annotations with laymen judgements
\end{enumerate}
\item Incentive for users: 
\begin{enumerate}
\item Estimate your political bias
\item Escape your political filter bubble
\end{enumerate}
\end{itemize}
\centering
\end{frame}


\begin{frame}\frametitle{Demo \url{http://rightornot.info}}
\centering 
\includegraphics[width=.9\textwidth]{images/active_learning_manifesto-left-neutral-right}\\
Labels: Left, Neutral, Right
\end{frame}

\begin{frame}\frametitle{Demo \url{http://rightornot.info}}
\centering 
\includegraphics[width=.6\textwidth]{images/web-demo}\\
\end{frame}

\begin{frame}\frametitle{References}
\usebeamerfont{bib}

\bibliographystyle{abbrvnat}
\def\newblock{}
\vspace{2em}
\bibliography{active-manifesto} 
\end{frame}

\end{document} 
